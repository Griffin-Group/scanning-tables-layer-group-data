\section{Scanning tables for the layer groups}

Herein we present tables listing the symmetry groups (penetration rod groups) which preserve lines penetrating through each layer group.
These tables scan through all possible locations and (in-plane rational) directions of the penetration lines, so are called scanning tables.

A machine-readable version of these tables, along with the source code and underlying data, is available at \cite{field_griffingroupscanning-tables-layer-group-data_2024}.

Each layer group has two tables with similar formats: one for high-symmetry directions, and one for oblique directions (an auxiliary table).
The elements of the tables are described below.

\subsection{Elements of the tables}

\subsubsection{Header}

Set above each table is the header.
It gives the Hermann-Mauguin (HM) symbol and the International Tables (IT) number of the layer group that is being scanned (the ``scanned group'').
The scanned group is always in its standard, default setting, as defined by the \textit{International Tables for Crystallography} volume E (\cite{kopsky_international_2010}).
Layer groups L52, L62, and L64 have the standard origin on the inversion centre (origin choice ``2'') rather than the 4-fold axis.

\subsubsection{Penetration direction}

The first column is the penetration direction.
It is the direction of the line penetrating through the rod group.
It is given by the integer indices $[uv0]$ and defines the basis vector $\mathbf{c}$, which will define the translation basis of the penetration rod group.

For the auxiliary tables, the penetration direction is grouped by whether $u$ and/or $v$ are odd or even.
$u$ and $v$ must also be co-prime, to ensure $\mathbf{c}$ is a primitive lattice vector.
Auxiliary tables of centred layer groups also add an extra column specifying whether the primitive basis vector $\mathbf{c}$ is $[u,v,0]$ or $[u/2,v/2,0]$ for a given choice of $u$ and $v$ (otherwise, it is assumed that $\mathbf{c}=[u,v,0]$).


\subsubsection{Scanning direction}

The second column is the scanning direction, given by the scanning vector $\mathbf{d}$.
Each scanning direction is paired with a penetration direction.
The primary role of $\mathbf{d}$ is to define the location of the penetration line, but it is also used to form a coordinate basis.
The scanning vector $\mathbf{d}$ is chosen such that $\mathbf{c}$ and $\mathbf{d}$ form a right-handed conventional basis for the scanning group.

For the auxiliary tables, the scanning direction is given by the integer indices $[pq0]$, with specified constraints on $p$ and $q$.
A conventional right-handed basis is ensured by solving for $(\mathbf{c}\times\mathbf{d})\cdot[001]=1$, and choosing $p,q$ to be co-prime.
For centred groups, $\mathbf{d}=[p/2,q/2,0]$ instead, with $(\mathbf{c}\times\mathbf{d})\cdot[001]=1/2$ ensuring a conventional basis.


\subsubsection{Scanning group}

The third column is the scanning group.
The scanning group is the maximal subgroup of the scanned group whose point group preserves the penetration direction.
By the scanning theorem (\cite{kopsky_scanning_1989}), the scanning table (specifically, the location and penetration rod group columns) of the scanned group along a particular direction is identical to the scanning table of the scanning group with the same setting and origin choice.

Each scanning group applies to all entries in the same row of the table, bounded by horizontal lines.
This may include multiple penetration directions and multiple locations.

The scanning group is a layer group.
Its HM symbol and IT number (prefixed by L for layer group) are given.
The basis is $(\mathbf{c},\mathbf{d},\mathbf{z})$, where $\mathbf{z}=[001]$ is an out-of-plane vector.
If the scanning group is not in the default setting, the IT number is marked by a prime as a convenience for the reader.
If the origin is not the conventional origin, then the position of the origin relative to the conventional origin is given in square brackets, in units of the scanning group basis.

\subsubsection{Location}

The fourth column is the location of the penetration line.
For points given by $P+s\mathbf{d}$, where $P$ is the scanned group origin, it gives a set of values $s$ in the unit interval $[0,1)$, with each row giving different penetration rod groups.

The first rows are special locations, with discrete values of $s$.
The last row for each scanning group is the general location, which is all values of $s$ not in a special location.

Locations are grouped using square brackets into orbits, that is, points which are the same under the operation of the scanning group.
If two values of $s$ are not bound by square brackets, then they are not in the same orbit.

\subsubsection{Penetration rod group}

The fifth column presents the penetration rod group for the given penetration line(s) specified by the location(s) and penetration direction(s).

The HM symbol and IT number (prefixed by R for rod group) are given.
If the rod group is not in the default setting, the IT number is marked by a prime as a convenience for the reader.
If the origin is not the conventional origin, then the position of the origin relative to the conventional origin in units of $\mathbf{c}$ is given in square brackets.

The sectional rod group is given in the basis $(\mathbf{d},\mathbf{z},\mathbf{c})$ with an origin $P+s\mathbf{d}$, where $P$ is the standard origin of the scanned group.
Note that, due to conventions for rod and layer groups, this is a different order of basis vectors to the scanning group.

The specific penetration rod group in the original basis is readily reconstructed from the rod group in the default basis and the information in the table using the transformation $Q g Q^{-1}$ for each element $g$ of the group in standard basis.
If the rod group is in its default setting, then the transformation matrix is
\begin{equation}
	Q = \left(\mathbf{d}\ \mathbf{z}\ \mathbf{c} | s\mathbf{d} + t\mathbf{c}\right),
\end{equation}
where $(A|\mathbf{b})$ is an affine transformation $y = Ax + \mathbf{b}$ and $t$ is the origin shift given in the table.
If the group is not in its default setting, then the transformation matrix is instead
\begin{equation}
	Q = \left(\mathbf{z}\ {-\mathbf{d}}\ \mathbf{c} | s\mathbf{d} + t\mathbf{c}\right).
\end{equation}


\begin{thebibliography}{23}
	\baselineskip=9pt\parskip=0pt
	
	\harvarditem{Field}{2024}{field_griffingroupscanning-tables-layer-group-data_2024}
	Field, B.  \harvardyearleft 2024\harvardyearright{}.
	\newblock {GriffinGroup}/scanning-tables-layer-group-data.
	\newline\harvardurl{https://doi.org/10.5281/zenodo.13948517}
	
	\harvarditem{Kopsk\'{y} \harvardand\ Litvin}{2010}{kopsky_international_2010}
	Kopsk\'{y}, V. \harvardand\ Litvin, D.~B. (eds.)  \harvardyearleft
	2010\harvardyearright{}.
	\newblock \emph{International {Tables} for {Crystallography}: {Subperiodic}
		groups}, vol.~E.
	\newblock Chester, England: International Union of Crystallography, 2nd ed.
	\newline\harvardurl{https://doi.org/10.1107/97809553602060000109}
	
	\harvarditem{Kopsk\'{y} \harvardand\ Litvin}{1989}{kopsky_scanning_1989}
	Kopsk\'{y}, V. \harvardand\ Litvin, D.~B.  \harvardyearleft
	1989\harvardyearright{}.
	\newblock In \emph{Group theoretical methods in physics}, edited by
	Y.~Saint-Aubin \harvardand\ L.~Vinet, pp. 263--266. Singapore: World
	Scientific.
\end{thebibliography}
